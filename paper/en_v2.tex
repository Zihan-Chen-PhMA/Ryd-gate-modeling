%!TEX program = pdflatex
\documentclass[aps,%
onecolumn,%
floatfix,%
pra,%
amsfonts,%
longbibliography,%
groupedaddress,%
superscriptaddress]{revtex4-2}%

%%%%%%%%%%%%%%%%%%%%%%%%%%%%%%%%%%%%%%%%%%%%%%%%%%%%%%%%%%%%%%%%%%%%%%%%%
% Packages
%%%%%%%%%%%%%%%%%%%%%%%%%%%%%%%%%%%%%%%%%%%%%%%%%%%%%%%%%%%%%%%%%%%%%%%%%
\usepackage{amsmath}
\usepackage{amssymb}
\usepackage{amsthm}
\usepackage{bm,bbm} % to use the command \mathbbm{1}, \bm{x}
\usepackage{graphicx}
\usepackage{indentfirst}
\usepackage{enumitem}
% \usepackage[style=numeric, backend=biber]{biblatex}
% \addbibresource{ref.bib}
% \usepackage{algorithmicx}
% \usepackage{algpseudocode}
% \usepackage{algorithm}
% \usepackage{algorithmic}
% \usepackage{algcompatible}
\usepackage{algpseudocode}
\usepackage{tikz}
\usepackage{quantikz}
\usetikzlibrary{graphs} 
\usepackage[titletoc,toc,page,title]{appendix}
\definecolor{beamer@blendedblue}{rgb}{0.2,0.2,0.7}      % color used in beamer
\usepackage[bookmarks=false]{hyperref}
\hypersetup{colorlinks=true,%
citecolor=beamer@blendedblue,%
linkcolor=beamer@blendedblue,%
urlcolor=beamer@blendedblue,%
pdfstartview=FitH,%
bookmarksopen=true}

\def\UrlBreaks{\do\/\do-} % url break
\usepackage[T1]{fontenc}
\usepackage{times}
\usepackage{booktabs} % to use the \toprule \midrule command
%%%%%%%%%%%%%%%%%%%%%%%%%%%%%%%%%%%%%%%%%%%%%%%%%%%%%%%%%%%%%%%%%%%%%%%%%
% General Commands
%%%%%%%%%%%%%%%%%%%%%%%%%%%%%%%%%%%%%%%%%%%%%%%%%%%%%%%%%%%%%%%%%%%%%%%%%
\newtheorem{theorem}{Theorem}
\newtheorem{proposition}[theorem]{Proposition}
\newtheorem{lemma}{Lemma}
\newtheorem{corollary}{Corollary}

%%%%%%%%%%%%%%%%%%%%%%%%%%%%%%%%%%%%%%%%%%%%%%%%%%%%%%%%%%%%%%%%%%%%%%%%%
\usepackage{pifont}

%%%%%%%%%%%%%%%%%%%%%%%%%%%%%%%%%%%%%%%%%%%%%%%%%%%%%%%%%%%%%%%%%%%%%%%%%
% Quantum Commands
%%%%%%%%%%%%%%%%%%%%%%%%%%%%%%%%%%%%%%%%%%%%%%%%%%%%%%%%%%%%%%%%%%%%%%%%%
\DeclareMathOperator{\Tr}{Tr}
\DeclareMathOperator{\tr}{Tr}
\DeclareMathOperator{\NOT}{NOT}
\DeclareMathOperator{\diag}{diag}
\DeclareMathOperator{\rank}{rank}
\DeclareMathOperator{\conv}{conv}
\DeclareMathOperator{\sgn}{sgn}
\DeclareMathOperator{\Sp}{Span}
\DeclareMathOperator{\id}{id}
% \newcommand{\bra}[1]{\langle #1\rvert}
% \newcommand{\ket}[1]{\lvert #1\rangle}
% \newcommand{\proj}[1]{\lvert #1\rangle\!\langle #1\rvert}
\newcommand{\ox}{\otimes}
\newcommand{\1}{\mathbbm{1}}
\newcommand{\mean}[1]{\langle #1\rangle}
\newcommand{\gen}[1]{\langle #1\rangle}
\newcommand{\ketbra}[2]{\lvert #1\rangle\!\langle #2\rvert}
% \newcommand{\braket}[2]{\langle #1\vert #2\rangle}
%%%%%%%%%%%%%%%%%%%%%%%%%%%%%%%%%%%%%%%%%%%%%%%%%%%%%%%%%%%%%%%%%%%%%%%%%
%%% Mathematical symbols
%%%%%%%%%%%%%%%%%%%%%%%%%%%%%%%%%%%%%%%%%%%%%%%%%%%%%%%%%%%%%%%%%%%%%%%%%
\newcommand*{\cF}{\mathcal{F}}
\newcommand*{\cH}{\mathcal{H}}
\newcommand*{\cO}{\mathcal{O}}
\newcommand*{\cP}{\mathcal{P}}
\newcommand*{\cS}{\mathcal{S}}
\newcommand*{\bF}{\mathbb{F}}
\newcommand*{\bP}{\mathbb{P}}

% Colors
\definecolor{alizarin}{rgb}{0.82, 0.1, 0.26}
\definecolor{googleblue}{HTML}{4285F4}
\definecolor{googlered}{HTML}{DB4437}
\definecolor{googleyellow}{HTML}{F4B400}
\definecolor{googlegreen}{HTML}{0F9D58}
\definecolor{klevinblue}{HTML}{002FA7}
\definecolor{tiffanyblue}{HTML}{0ABAB5}

\newcommand{\KW}[1]{{\color{googlered}(KW: #1)}}
\newcommand{\SYC}[1]{\textcolor{blue}{(SYC: #1)}}

\begin{document}

\title{Simulation of Rydberg protocol}
%%%
\author{Siyuan Chen}
\affiliation{Hefei National Research Center for Physical Sciences at the Microscale and School of Physical Sciences, 
University of Science and Technology of China, Hefei 230026, China}%
%%%


\begin{abstract}
\noindent 

\end{abstract}
\date{\today}
\maketitle

\section{Preliminaries}
In this section, we give the real simulation of $\text{CPHASE}$ gate used by Evered \emph{et.al.}~\cite{Evered_2023}, based on $^{87}Rb$ atoms.
\subsection{Experimental setting}

\begin{figure}[!htbp]
    \centering
    \includegraphics[width=0.5\linewidth]{Real_Rb_Elevel.png}
    \caption{Related energy level of $^{87}{Rb}$. Atoms is excited to Rydberg state through two lasers with $1013\text{nm}$ wave-length and $420\text{nm}$ wave-length. }
    \label{fig:Real_Rb_Elevel}
\end{figure}
We sequentially choose the basis states $\ket{0}: \{5S_{1/2}, F = 1, m_F = 0\}$ and $\ket{1}: \{5S_{1/2}, F = 2, m_F = 0\}$ as the qubit space. The intermediate states, obtained through $420\text{nm}$  $\sigma_-$ light with large blue-detuning $\Delta = 6.1 \text{GHz}$, consist of three levels split by hyperfine structure:
\[\ket{2}, \ket{3}, \ket{4} = \{6P_{3/2}, F = 1, 2, 3, m_F = -1\}\]
For Rydberg levels, the dominant term in the electronic Hamiltonian is $\vec{B} \cdot (\vec{L} + \vec{S})$, far greater than the coupling term of nuclear spin $\vec{I} \cdot (\vec{L} + \vec{S})$. Viewing nuclear spin coupling as a perturbation, the unperturbed Hamiltonian commutes with $L^2$, $S^2$, $J^2$, and $J_z$, hence the choice of the $m_j$ representation.
We aim to excite atoms to $\ket{5} = \{70S_{1/2}, m_j = -1/2\}$ using $1013\text{nm}$ $\sigma_+$ light. However, the intermediate state $\{6P_{3/2}, F = 1, 2, 3, m_F = -1\}$ have components $\{6P_{3/2}, m_j = -3/2\}$ and components $\{6P_{3/2}, m_j = -1/2\}$, atoms will inevitably be excited to $\ket{6} = \{70S_{1/2}, m_j = +1/2\}$. In the experiment, a magnetic field is applied to induce a $24$ MHz splitting between $\ket{5}$ and $\ket{6}$ to mitigate this effect.


\subsection{The derivation of the Hamiltonian}
In protocols of Control-Z gate, the decreasing of the gate time will not only speed up the computation but also mitigate the decay error on the Rydberg states. 
However, For most one-photon process of the the Rydberg excitation, the frequency of the laser is high thus hard to increase the intensity. 
Considering one atom with energy level $\{\omega_{r_j}\}_{j =0}^{m_r}, \{\omega_{e_j}\}_{j = 0}^{m_e}$ and $\{\omega_{g_j}\}_{j = 0}^{m_g}$ and eigenstate
 $\{\ket{r_j}\}_{j =0}^{m_r}, \{\ket{e_j}\}_{j = 0}^{m_e}$ and $\{\ket{g_j}\}_{j = 0}^{m_g}$. Two laser light has electric vector $\mathbf{E}^{(1)}(t)$ and $\mathbf{E}^{(2)}(t)$ with 
 \begin{align}
    \mathbf{E}^{(j)}(t) = \left({\mathbf{E}^{(j)}_-}/{2}\right) e^{i \left(\omega^{(j)} t + \phi^{(j)}(t)\right)} + \left({\mathbf{E}_+^{(j)}}/{2}\right)e^{-i\left(\omega^{(j)} t + \phi^{(j)}(t)\right) },
 \end{align}
 where $\mathbf{E}^{(j)}_- =E^{(j)} \cdot\{\mathbf{e}_x + i \mathbf{e}_y,\mathbf{e}_x - i \mathbf{e}_y,\mathbf{e}_z\}$ corresponds to three types of polarization $\{\sigma_{-}, \sigma_{+}, \sigma_{z}\}$. The Hamiltonian then can be written as:
\begin{align}
    H &= H_0  + e\mathbf{r} \cdot \mathbf{E}^{(1)}(t) + e\mathbf{r} \cdot \mathbf{E}^{(2)}(t)\\
    &= \sum_{j = 0}^{m_r}\hbar \omega_{r_j} \proj{r_j} + \sum_{j = 0}^{m_e}\hbar \omega_{e_j} \proj{e_j} +\sum_{j = 0}^{m_g}\hbar \omega_{g_j} \proj{g_j}  \\
    &+ \left(\sum_{j = 0}^{m_{g}}\sum_{jk= 0}^{m_e}\hbar \Omega_{jk,-}^{(1)}(t)/2 \exp[i(\omega_{e_0} - \omega_{g_0} -\Delta)t + i\phi_1(t)] \right)\ket{g_j}\bra{e_k} + c.c. \\
    &+ \left(\sum_{j = 0}^{m_{g}}\sum_{jk= 0}^{m_e}\hbar \Omega_{jk,+}^{(1)}(t)/2 \exp[-i(\omega_{e_0} - \omega_{g_0} -\Delta)t - i\phi_1(t)] \right)\ket{g_j}\bra{e_k} + c.c. \\
    &+\left(\sum_{j = 0}^{m_{e}}\sum_{jk= 0}^{m_r}\hbar \Omega^{(2)}_{jk,-}(t)/2 \exp[i(\omega_{r_0} - \omega_{e_0} +\Delta -\delta)t + i\phi_2(t)] \right)\ket{e_j}\bra{r_k} + c.c.\\
    &+\left(\sum_{j = 0}^{m_{e}}\sum_{jk= 0}^{m_r}\hbar \Omega^{(2)}_{jk,+}(t)/2 \exp[-i(\omega_{r_0} - \omega_{e_0} +\Delta -\delta)t - i\phi_2(t)] \right)\ket{e_j}\bra{r_k} + c.c.
    \label{eq:two_photon_H}
\end{align}
Here we use $\omega^{(1)} = \omega_{e_0} - \omega_{g_0} - \Delta$ and $\omega^{(2)} =  \omega_{r_0} - \omega_{e_0} + \Delta - \delta$ to represent two red-detuning laser frequency.
The parameter $\Omega^{(1,2)}_{jk,\pm}$ are decided by
\begin{align}
    \Omega^{(1)}_{jk,\pm} = \bra{g_j}e\mathbf{r} \cdot \mathbf{E}^{(1)}_{\pm}\ket{e_k}/\hbar,\quad \Omega^{(2)}_{jk,\pm} = \bra{e_j}e\mathbf{r} \cdot \mathbf{E}^{(2)}_{\pm}\ket{r_k}/\hbar.
\end{align}
Consider the time-dependent unitary transition 
\begin{align}
    U(t) = \exp\left[{ i\omega_{g_0} t \sum_{j = 0}^{m_g} \proj{g_j} + i(\omega_{e_0} - \Delta)t \sum_{j = 0}^{m_e}\proj{e_j} + i(\omega_{r_0} - \delta)t \sum_{j = 0}^{m_r}\proj{r_j}}\right].
\end{align}
Using the rotating wave approximation, we ignore the fast oscillating term and gives the Hamiltonian:
\begin{align}
\label{eq:Lamman-two-photon}
    &H' = i \hbar \frac{dU}{dt} U^{\dagger} + UHU^\dagger\\
    &= \sum_{j = 0}^{m_r}\hbar (\delta + \omega_{r_j} - \omega_{r_0}) \proj{r_j} + \sum_{j = 0}^{m_e}\hbar (\Delta + \omega_{e_j} - \omega_{{e_0}}) \proj{e_j} + \sum_{j = 0}^{m_g}\hbar(\omega_{g_j} - \omega_{g_0})\proj{g_j}\\
    &\quad+ \sum_{j = 0}^{m_{g}}\sum_{jk= 0}^{m_e}\frac{\hbar \Omega_{jk,-}^{(1)}(t) e^{i \phi_1(t)}}{2} \ket{g_j}\!\bra{e_k} + c.c. + \sum_{j = 0}^{m_{e}}\sum_{jk= 0}^{m_r}\frac{\hbar \Omega^{(2)}_{jk,-}(t) e^{i \phi_2(t)}}{2} \ket{e_j}\!\bra{r_k} + c.c.~.
\end{align}
For two atoms inside the Rydberg radius, we add the Rydberg interaction Hamiltonian 
\begin{align}
    H'_{\rm{int}} &=\sum_{j,k = 0}^{m_r} C_6(jk)\cdot R^{-6} \ket{r_jr_k}\bra{r_jr_k}.
\end{align}

In the case of $^{87}Rb$, we imposed two laser light $E^{(1)}, \sigma_{-}$ with wavelength $420~\mathrm{nm}$ and $E^{(2)},\sigma_{+}$ with wavelength $1013~\mathrm{nm}$.
 We keep the intensity $E^{(1)},E^{(2)}$ of both the laser unchanged and allow $420~nm$ light change its phase $\phi^{(1)}(t)$.
According to the $CG$-coefficient, we are able to transfer from the hyperfine-basis into $\{L^2,S^2,J^2,I^2,m_J,m_I\}$ basis.
\begin{align*}
\ket{0} = \ket{g_1} &=(5S_{1/2}) 1/\sqrt{2}\left[-\{m_j = 1/2, m_I = -1/2\} + \{ m_j = -1/2, m_I = 1/2\}\right], \\
\ket{1} = \ket{g_0} &=(5S_{1/2}) 1/\sqrt{2}\left[\{5S_{1/2}, m_j = 1/2, m_I = -1/2\} +\{5S_{1/2}, m_j = -1/2, m_I = 1/2\}\right], \\
\ket{2} = \ket{e_1} &=(6P_{3/2}) \sqrt{\frac{3}{10}}\{m_j = -3/2, m_I = 1/2\} -\sqrt{\frac{2}{5}}\{m_j = -1/2, m_I = -1/2\} +\sqrt{\frac{3}{10}} \{m_j = 1/2, m_I = -3/2\},\\
\ket{3} = \ket{e_0} &=(6P_{3/2}) \frac{1}{\sqrt{2}}\{m_j = -3/2, m_I = 1/2\} -\frac{1}{\sqrt{2}} \{m_j = 1/2, m_I = -3/2\}, \\
\ket{4} = \ket{e_2} &=(6P_{3/2}) \frac{1}{\sqrt{5}}\{m_j = -3/2, m_I = 1/2\} +\sqrt{\frac{3}{5}}\{m_j = -1/2, m_I = -1/2\} +\frac{1}{\sqrt{5}} \{m_j = 1/2, m_I = -3/2\},\\
\ket{5} = \ket{r_0} &= (70S_{1/2}) \{m_j = -1/2,m_I = 1/2\},\quad  \ket{6}=\ket{r_1} = (70S_{1/2}) \{m_j = 1/2,m_I = -1/2\}.
\end{align*}
We labeled $\Omega_{420}$ as the coupling coefficient:
\begin{align}
    \Omega_{420} =: {\langle \{6P_{3/2}, m_j = -3/2, m_I = 1/2\}| e\mathbf{r}\cdot \mathbf{E}_-^{(1)}|\{5S_{1/2}, m_j = -1/2, m_I = 1/2\} \rangle}/{\sqrt{2}\hbar},\\
    \gamma \Omega_{420} = {\langle \{6P_{3/2}, m_j = -1/2, m_I = -1/2\} | e\mathbf{r}\cdot\mathbf{E}_-^{(1)} | \{5S_{1/2}, m_j = 1/2, m_I = -1/2\}\rangle}/{\sqrt{2}\hbar} .
\end{align}
Using the notation $\text{CG}(j_1,m_1,j_2,m_2,j_3,m_3) = \langle j_1,m_1, j_2,m_2\ket{j_1,j_2,j_3,m_3}$, we can further express:
\begin{align*}
\langle 2| H | 1 \rangle &= \text{CG}(3/2,-3/2,3/2,1/2,1,-1)\Omega_{420}/2 +\text{CG}(3/2,-1/2,3/2,-1/2,1,-1) \gamma \Omega_{420}/2 \\
&=\sqrt{0.3}\Omega_{420}/2 - \sqrt{0.4}\gamma \Omega_{420}/2 \\
\langle 3| H | 1 \rangle &= \text{CG}(3/2,-3/2,3/2,1/2,2,-1)\Omega_{420}/2 +\text{CG}(3/2,-1/2,3/2,-1/2,2,-1) \gamma \Omega_{420}/2 \\
&=-\sqrt{0.5}\Omega_{420}/2 \\
\langle 4| H | 1 \rangle &= \text{CG}(3/2,-3/2,3/2,1/2,3,-1)\Omega_{420}/2 +\text{CG}(3/2,-1/2,3/2,-1/2,3,-1) \gamma \Omega_{420}/2 \\
&=\sqrt{0.2}\Omega_{420}/2 + \sqrt{0.6}\gamma \Omega_{420}/2 
\end{align*}
Similarly, the action of $1013~\mathrm{nm}$-laser can be expressed analogously. Define:
\begin{align}
\Omega_{1013} &=: \langle\{70S_{1/2},m_j = -1/2,,m_I = 1/2\}| e\mathbf{r}\cdot \mathbf{E}_-^{(2)} |\{6P_{3/2},m_j = -3/2,m_I = 1/2\}\rangle/\hbar, \\
\beta \Omega_{1013} &=: \langle\{70S_{1/2},m_j = 1/2,m_I = -1/2\}| e\mathbf{r}\cdot \mathbf{E}_-^{(2)}|\{6P_{3/2},m_j = -1/2,,m_I = -1/2\}\rangle/\hbar,
\end{align}
and we get the result
\begin{align*}
\langle 5| H | 2 \rangle &= \text{CG}(3/2,-3/2,3/2,1/2,1,-1)\Omega_{1013}/2 = \sqrt{0.3}\Omega_{1013}/2   \\
\langle 5| H | 3 \rangle &= \text{CG}(3/2,-3/2,3/2,1/2,2,-1)\Omega_{1013}/2= -\sqrt{0.5}\Omega_{1013}/2   \\
\langle 5| H | 4 \rangle &= \text{CG}(3/2,-3/2,3/2,1/2,3,-1)\Omega_{1013}/2= \sqrt{0.2}\Omega_{1013}/2  \\
\langle 6| H | 2 \rangle &= \text{CG}(3/2,-1/2,3/2,-1/2,1,-1) \beta \Omega_{1013}/2= -\sqrt{0.4}\beta\Omega_{1013}/2  \\
\langle 6| H | 3 \rangle &= \text{CG}(3/2,-1/2,3/2,-1/2,2,-1) \beta \Omega_{1013}/2= 0  \\
\langle 6| H | 4 \rangle &= \text{CG}(3/2,-1/2,3/2,-1/2,3,-1) \beta \Omega_{1013}/2= \sqrt{0.6}\beta\Omega_{1013}/2 \\
\end{align*}
The specific value of $\gamma,\beta$ can be obtained by integrating the wave function. Here, we directly use \verb|Arc| to obtain the result: $\gamma = \beta = \sqrt{1/3}$.
For simplicity,  we denote $\bra{i}H\ket{j} =: H_{ij}$. The process in Figure~\ref{fig:Real_Rb_Elevel} is stimulated Raman transitions concerning three intermediate states $\{\ket{2},\ket{3},\ket{4}\}$. Noted that due to the detuning between two Rydberg states, we ignore the influence of state $\ket{6}$. We then write the Schrodinger equation according to Eq.~\eqref{eq:Lamman-two-photon}.
\begin{align}
    i\hbar \partial_t \psi_i &= (H_{i1} \psi_1 + H_{i5}\psi_5) + \Delta \psi_i \quad \forall~i = 2,3,4, \\ 
    i\hbar\partial_t \psi_5 &= \sum_{i = 2,3,4} H_{5i} \psi_i,\\
    i\hbar\partial_t \psi_1 &=  \sum_{i = 2,3,4} H_{1i} \psi_i.
\end{align}
We can adiabatically eliminate the intermediate states through setting $\partial_t \psi_i = 0,~\forall~i = 3,4,5$. This then give the effective two-level transition between state $\ket{1}$ and state $\ket{5}$.
\begin{align}
    i\hbar\partial_t \psi_5&=  -\sum_{i = 2,3,4} \frac{|H_{5i}|^2}{\Delta} \psi_5 - \left(\sum_{i = 2,3,4} \frac{H_{i1} H_{5i}}{\Delta}\right) \psi_1  ,\\
      i\hbar\partial_t \psi_1 &=  -\sum_{i = 2,3,4} \frac{|H_{1i}|^2}{\Delta} \psi_1 - \left(\sum_{i = 2,3,4} \frac{H_{i5} H_{1i}}{\Delta}\right) \psi_5  .
\end{align}
Then the effective light-shift and the effective Rabi frequency write:
\begin{align}
    \Delta_{\text{eff,5}} &= -\sum_{i = 2,3,4} \frac{|H_{5i}|^2} {\Delta} = -\frac{\Omega_{1013}^2}{4\Delta},\\
     \Delta_{\text{eff,1}} &= -\sum_{i = 2,3,4} \frac{|H_{1i}|^2} {\Delta} =-\left[ (\sqrt{0.3} - \sqrt{\frac{0.4}{3}})^2+(\sqrt{0.2} + \sqrt{\frac{0.6}{3}})^2 +0.5\right]\frac{\Omega_{420}^2}{4\Delta} =- \frac{4}{3} \frac{\Omega_{420}^2}{4\Delta},\\
    \Omega_{\text{eff}} &= -2\frac{\sum_{i = 2,3,4} H_{i1} H_{5i}}{\Delta} = -\frac{\Omega_{1013}\Omega_{420}}{2\Delta}.
\end{align}

\section{Results}
Then we test the phase shift that is robust to amplitude noise.
This optimized function can be fitted with two trigonometric functions as
\begin{align}
    \varphi(t) = A_1 \sin(\omega t + \phi_1) + A_2 \cdot \sin(2 \omega t + \phi_2) + \delta t,
\end{align}
with $\omega = 8.71374888 \times 10^{-1} \Omega_{\mathrm{eff}}, A_1 = 8.89646230 \times 10^{-1}, \phi_1 = -8.93893845\times 10^{-1}, A_2 = -6.67698078 \times 10^{-1}, \phi_2 = 1.35391702, \delta =  -1.97859545 \times 10^{-2} \Omega_{\mathrm{eff}}$. The gate time is chosen as $t_{\mathrm{gate}} = 2.62170756 \cdot ({2 \pi}/{\Omega_{\mathrm{eff}}})$. 




\appendix
\section{Estimates Rabi-frequency from light shift}
The intensity of the laser light, in other words, the value of $\Omega$.
Noticed that when the laser frequency heavily deviates all of the energy level 

\section{Atom Decay\label{app:atom_decay}}
The Hamiltonian with dual energy levels $\ket{g},\ket{e}$ and single-mode light field (frequency $\omega$) coupled is:
\begin{align}
\mathcal{V} &= \mathbb{P} e \vec{r} \cdot\vec{E} \mathbb{P} \\
&= e  (\langle e|\vec{r}|g\rangle\sigma_- + \langle g|\vec{r}|e\rangle\sigma_+) \cdot [\sqrt{\frac{\hbar \omega }{2\varepsilon_0}}(\vec{f}(\vec{r}) a + \vec{f}^*(\vec{r}) a^\dagger)].
\end{align}
Using chiral symmetry, it is proved that $\langle g|\vec{r}|g\rangle = \langle e|\vec{r}|e\rangle = 0$. Under the interaction representation, the rotation wave approximation can be obtained:
\begin{align}
    \mathcal{V}(t) = \hbar  [g(r) a^\dagger \sigma_{-} e^{-i\delta t} + g^*(r)a \sigma_{+}e^{i\delta t} ],\quad  \delta = \Delta E/\hbar-\omega,
\end{align}
where $g(r) = e\sqrt{\frac{\hbar \omega}{2\varepsilon_0}} \vec{f}(\vec{r}) \cdot \langle g|\vec{r}|e\rangle$. For multi-mode light fields, similarly, the coupling Hamiltonian becomes:
\begin{align}
    \mathcal{V}(t) = \sum_{k,\zeta} \hbar  [g_{k,\zeta}(\vec{r})a_{k,\zeta}^\dagger \sigma_{-} e^{-i\delta_{k} t} + g^*_{k,\zeta}(\vec{r})a_{k,\zeta} \sigma_{+}e^{i\delta_{k} t} ],\quad  \delta_{k} = \Delta E/\hbar-\omega_{k},
\end{align}
where $g_{k,\zeta}(\vec{r}) = e\sqrt{\frac{\hbar \omega_k}{2\varepsilon_0}} \vec{f}_{k,\zeta}(\vec{r}) \cdot \langle g|\vec{r}|e\rangle$. Assuming that the thermal reservoir is large enough and the coupling with atoms is weak enough, the equation for the evolution of the atomic density matrix can be obtained~\cite{Quantumoptics}:
\begin{align}\label{eq:weak-int-reservoir-evol}
\dot\rho_{a}(t) = -\frac{i}{\hbar}Tr_{R} [\mathcal{V}(t), \rho_a(t_i)\otimes\rho_R(t_i)] -\int_{t_i}^t dt' \frac{1}{\hbar^2}Tr_{R} [\mathcal{V}(t), [\mathcal{V}(t'), \rho_a(t)\otimes\rho_R(t_i)]]
\end{align}
For the first parts, we see
\begin{align} 
-\frac{i}{\hbar} Tr_{R}[\mathcal{V}(t), \rho_a(t_i)\otimes\rho_R(t_i)] &=-i \sum_{k,\zeta}  Tr_{R}[g_{k,\zeta}(\vec{r})a_{k,\zeta}^\dagger \sigma_{-} e^{-i\delta_{k} t} + c.c. , \rho_a(t_i)\otimes\rho_R(t_i)] \\
&=-i \sum_{k,\zeta}  Tr_{R}[g_{k,\zeta}(\vec{r})a_{k,\zeta}^\dagger \sigma_{-} e^{-i\delta_{k} t}  , \rho_a(t_i)\otimes\rho_R(t_i)]+c.c. \\
&=-i \sum_{k,\zeta} g_{k,\zeta}(\vec{r})\langle a_{k,\zeta}^\dagger \rangle  e^{-i\delta_{k} t} [\sigma_{-} , \rho_a(t_i)]+c.c. 
\end{align}
For the second integral, we simplified it as:
\begin{align}
 &\frac{1}{\hbar^2}Tr_{R} [\mathcal{V}(t), [\mathcal{V}(t'),\rho_a(t)\otimes\rho_R(t_i)]] \\
&= \frac{1}{\hbar} \sum_{\substack{k',\zeta'}}Tr_{R} [ \mathcal{V}(t), [g_{k',\zeta'}(\vec{r})a_{k',\zeta'}^\dagger \sigma_{-} e^{-i\delta_{k'} t'} +c.c.\ , \rho_a(t)\otimes\rho_R(t_i)]]\\
&= \frac{1}{\hbar}\sum_{\substack{k',\zeta'}}g_{k',\zeta'} (\vec{r})e^{-i\delta_{k'} t'}Tr_{R} [ \mathcal{V}(t), [a_{k',\zeta'}^\dagger \sigma_{-} , \rho_a(t)\otimes\rho_R(t_i)] ] + c.c.\\
&=\sum_{\substack{k',\zeta'\\k.\zeta}}g_{k',\zeta'} (\vec{r})e^{-i\delta_{k'} t'}Tr_{R} [g_{k,\zeta}(\vec{r})a_{k,\zeta}^\dagger \sigma_{-} e^{-i\delta_{k} t} + g^*_{k,\zeta}(\vec{r})a_{k,\zeta} \sigma_{+}e^{i\delta_{k} t} , [a_{k',\zeta'}^\dagger \sigma_{-} , \rho_a(t)\otimes\rho_R(t_i)] ] + c.c.\\
&=\sum_{\substack{k',\zeta'\\k.\zeta}}g_{k',\zeta'} (\vec{r})e^{-i\delta_{k'} t'}g_{k,\zeta}(\vec{r})e^{-i\delta_{k} t}Tr_{R} [a_{k,\zeta}^\dagger \sigma_{-}  , [a_{k',\zeta'}^\dagger \sigma_{-} , \rho_a(t)\otimes\rho_R(t_i)] ] \\
&\quad\quad\quad+ g_{k',\zeta'} (\vec{r})e^{-i\delta_{k'} t'}g^*_{k,\zeta}(\vec{r})e^{i\delta_{k} t}Tr_{R} [a_{k,\zeta} \sigma_{+} , [a_{k',\zeta'}^\dagger \sigma_{-} , \rho_a(t)\otimes\rho_R(t_i)] ] + c.c.\\
&=\sum_{\substack{k',\zeta'\\k.\zeta}} -g_{k',\zeta'} (\vec{r})g_{k,\zeta}(\vec{r})e^{-i\delta_{k'} t'-i\delta_{k} t} \langle a_{k,\zeta}^\dagger a_{k',\zeta'}^\dagger \rangle  \times 2\sigma_{-}\rho_a(t) \sigma_{-} \\
&\quad\quad\quad+ g_{k',\zeta'} (\vec{r})g^*_{k,\zeta}(\vec{r})e^{i\delta_{k} t-i\delta_{k'} t'} \langle a_{k,\zeta} a_{k',\zeta'}^\dagger \rangle [\sigma_{+}\sigma_{-}\rho_a(t) - \sigma_{-}\rho_a(t)\sigma_{+}  ]\\
 &\quad\quad\quad+ g_{k',\zeta'} (\vec{r})g^*_{k,\zeta}(\vec{r})e^{i\delta_{k} t-i\delta_{k'} t'} \langle a_{k',\zeta'}^\dagger a_{k,\zeta}  \rangle  [\rho_a(t)\sigma_{-}\sigma_{+} - \sigma_{+}\rho_a(t)\sigma_{-}]+ c.c.
\end{align}
The photons reservoir coupled to atoms is a canonical ensemble with temperature $T$, and its density matrix is:
\begin{align}
\rho_R = \frac{1}{Z} e^{-\beta H} = \frac{1}{Z} e^{-\beta \sum_{k,\zeta}\hbar \omega_k a_{k,\zeta}^\dagger a_{k,\zeta}} 
\end{align}
Using the condition that the trace of the density matrix is 1, $Z$ can be obtained as follows:
\begin{align}
Z = Tr\ e^{\sum_{k,\zeta}-\beta \hbar \omega_k a_{k,\zeta}^\dagger a_{k,\zeta}} &= \sum_{array~n_{k,\zeta} }\langle n_{k,\zeta} |e^{\sum_{k,\zeta}-\beta \hbar \omega_k a_{k,\zeta}^\dagger a_{k,\zeta}} | n_{k,\zeta} \rangle = \sum_{array~n_{k,\zeta} } \prod_{k,\zeta}e^{-\beta \hbar \omega_k n_{k,\zeta}} \\
&=\prod_{k,\zeta}\sum_{n_{k,\zeta}=0 }^{\infty}e^{\sum_{k,\zeta} -\beta \hbar \omega_k n_{k,\zeta}} = \prod_{k,\zeta} \frac{1}{1 -e^{-\beta \hbar \omega_k }  }
\end{align} 
We could also calculates non-vanish average as below:
\begin{align}
Tr[\rho_R a_{k_1,\zeta_1}^\dagger a_{k_2,\zeta_2}] &= Z^{-1}\sum_{array~n_{k,\zeta} }\langle n_{k,\zeta} |e^{\sum_{k,\zeta}-\beta \hbar \omega_k a_{k,\zeta}^\dagger a_{k,\zeta}} a_{k_1,\zeta_1}^\dagger a_{k_2,\zeta_2}| n_{k,\zeta} \rangle \\
&=Z^{-1}  \delta_{k_1k_2,\zeta_1\zeta_2}\sum_{array~n_{k,\zeta} } e^{\sum_{k,\zeta}-\beta \hbar \omega_k n_{k,\zeta}} n_{k_2,\zeta_2} \\
&=\delta_{k_1k_2,\zeta_1\zeta_2} \frac{1}{e^{\beta\hbar\omega_{k_2}} - 1} := \delta_{k_1k_2,\zeta_1\zeta_2} \bar{n}_{k_2,\zeta_2}.
\end{align}
\begin{align}
Tr[\rho_R  a_{k_2,\zeta_2}a_{k_1,\zeta_1}^\dagger] &= Tr[\rho_R  a_{k_1,\zeta_1}^\dagger a_{k_2,\zeta_2}] + Tr\{\rho_R  [a_{k_2,\zeta_2}, a_{k_1,\zeta_1}^\dagger] \} \\
&=\delta_{k_1k_2,\zeta_1\zeta_2}( \bar{n}_{k_2,\zeta_2}+1).
\end{align}
Plug into Eq.~\eqref{eq:weak-int-reservoir-evol}, we see that:
\begin{align}
\dot\rho_{a} &= -\int_{t_i}^t dt'\{\sum_{\substack{k,\zeta}}  g_{k,\zeta} (\vec{r})g^*_{k,\zeta}(\vec{r})e^{i\delta_{k} (t- t')} ( \bar{n}_{k,\zeta}+1) [\sigma_{+}\sigma_{-}\rho_a(t) -\sigma_{-}\rho_a(t)\sigma_{+} ]\\
 &\quad\quad\quad+ g_{k,\zeta} (\vec{r})g^*_{k,\zeta}(\vec{r})e^{i\delta_{k} (t- t')} \bar{n}_{k,\zeta}[\rho_a(t)\sigma_{-}\sigma_{+} -  \sigma_{+}\rho_a(t)\sigma_{-} ]\}+ c.c.
\end{align}
By introducing the definition of $g_{k,\zeta}$, the equation can be further simplified:
\begin{align}
\sum_{\substack{k,\zeta}}  g_{k,\zeta} (\vec{r})g^*_{k,\zeta}(\vec{r}) &= \sum_{\substack{k,\zeta}}e^2 \frac{\hbar \omega_k}{2\varepsilon_0} |f_{k,\zeta}(\vec{r})|^2 \cdot \varepsilon^i_{k,\zeta}\varepsilon^{j*}_{k,\zeta}\cdot \langle g|r^i|e\rangle  \langle e|r^j|g\rangle 
\end{align}
For light field in free space, $f_{k,\zeta}(\vec{r}) = \frac{1}{\sqrt{V}}e^{i\vec{k}\cdot\vec{r }}$, Moreover, $\sum_{\zeta}\varepsilon^i_{k,\zeta}\varepsilon^{j*}_{k,\zeta} = \delta^{ij} - \hat k^i \hat k ^j$. Then:
\begin{align}
\sum_{\substack{k,\zeta}}  g_{k,\zeta} (\vec{r})g^*_{k,\zeta}(\vec{r}) &= \sum_{k}e^2 \frac{\hbar \omega_k}{2\varepsilon_0} \frac{1}{V} \cdot (\delta^{ij} - \hat k^i \hat k ^j)\cdot \langle g|r^i|e\rangle  \langle e|r^j|g\rangle \\
&=\frac{V}{(2\pi)^3}\int d^3k ~e^2 \frac{\hbar \omega_k}{2\varepsilon_0} \frac{1}{V} \cdot (\delta^{ij} - \hat k^i \hat k ^j)\cdot \langle g|r^i|e\rangle  \langle e|r^j|g\rangle \\
&=\frac{1}{(2\pi)^3}\int dk d\theta ~ 2\pi k^2 \sin\theta ~e^2 \frac{\hbar \omega_k}{2\varepsilon_0}  \cdot (1 - \cos^2\theta)\cdot |\langle g|\vec{r}|e\rangle|^2 \\
&=\frac{4}{3}\frac{1}{(2\pi)^2} |\langle g|e\vec{r}|e\rangle|^2\frac{\hbar c }{2\varepsilon_0}\int dk  k^3   = \frac{\hbar |\langle g|e\vec{r}|e\rangle|^2}{6 \pi^2 \varepsilon_0 c^3}\int d\omega  \omega^3 
\end{align}
We see that $\lim_{t \to \infty}\int_{t_i}^t dt' e^{i\delta_k (t - t')} \approx \pi \delta(\omega_k - \omega_0) +i\times p.v. (\frac{1}{\omega_0 - \omega}), \omega_0 = \Delta E/\hbar$, neglecting the imaginary term, which is one of the contribution to the \textbf{Lamb shifts} after appropriate renormalization. Then
\begin{align}
\dot\rho_{a} &= -\frac{\hbar |\langle g|e\vec{r}|e\rangle|^2\omega_0^3 }{6 \pi \varepsilon_0 c^3}   ( \bar{n}_{k_0,\zeta}+1) [\sigma_{+}\sigma_{-}\rho_a(t) -\sigma_{-}\rho_a(t)\sigma_{+}  ]\\
 &\quad\quad\quad- \frac{\hbar |\langle g|e\vec{r}|e\rangle|^2\omega_0^3 }{6 \pi \varepsilon_0 c^3}  \bar{n}_{k_0,\zeta}[\sigma_{-}\sigma_{+}\rho_a(t) - \sigma_{+}\rho_a(t)\sigma_{-} ]+ c.c. \\
&=-(\Gamma/2)   ( \bar{n}_{k_0,\zeta}+1) [\sigma_{+}\sigma_{-}\rho_a(t) -  \sigma_{-}\rho_a(t)\sigma_{+} ]- (\Gamma/2)  \bar{n}_{k_0,\zeta}[\rho_a(t)\sigma_{-}\sigma_{+} -\sigma_{+}\rho_a(t)\sigma_{-} ]+ c.c.,
\end{align}
where $\Gamma = \frac{\hbar |\langle g|e\vec{r}|e\rangle|^2\omega_0^3 }{3 \pi \varepsilon_0 c^3} $. 


\end{document}